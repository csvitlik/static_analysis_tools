\documentclass[12pt]{article}

\usepackage[margin=1in]{geometry}
\renewcommand{\sfdefault}{phv}
\renewcommand{\rmdefault}{phv}
\renewcommand{\ttdefault}{pcr}
\usepackage[T1]{fontenc}
\usepackage[utf8]{inputenc}
\usepackage{hyperref}
\usepackage{xcolor}
\usepackage{titlesec}

\setlength{\parindent}{0pt}

\definecolor{Blue}{rgb}{0.2 0.5 1.0}

\hypersetup{citecolor=Blue,%
            colorlinks=true,%
            linkcolor=Blue,%
            pdfborder=0 0 0,%
            urlcolor=Blue,%
}

\let\myurl\url%
\renewcommand{\url}[1]{\hfill \\ \textbf{\myurl{#1}}%
}
\let\mytexttt\texttt%
\renewcommand{\texttt}[1]{\textbf{\mytexttt{#1}}%
}

\titleformat*{\section}{compact}
\titleformat{\section}{\Large\bfseries}{}{0pt}{}{}

\titlespacing{\section}{0pt}{10pt}{10pt}

\title{Static Analysis Tools}

\begin{document}

\maketitle
\tableofcontents
% {{{ AddressSanitizer
\section{AddressSanitizer}
\begin{quote}
AddressSanitizer (aka ASan) is a memory error detector for C/C++. It finds:

\begin{itemize}
    \item Use after free (dangling pointer dereference)
    \item Heap buffer overflow
    \item Stack buffer overflow
    \item Global buffer overflow
    \item Use after return
    \item Initialization order bugs
    \item Memory leaks
\end{itemize}
\end{quote}

\label{asan}
This, in combination with \hyperref[msan]{other} \hyperref[tsan]{sanitizers},
allows for powerful debugging capabilities.

\cite{asan}
% }}}
% {{{ AFL
\section{American Fuzzy Lop}
\begin{quote}
American fuzzy lop is a security-oriented fuzzer that employs a
novel type of compile-time instrumentation and genetic algorithms to
automatically discover clean, interesting test cases that trigger new
internal states in the targeted binary. This substantially improves
the functional coverage for the fuzzed code. The compact synthesized
corpora produced by the tool are also useful for seeding other,
more labor- or resource-intensive testing regimes down the road.
\end{quote}

Very useful tool for testing an extensive amount of random inputs.

\cite{afl}
% }}}
% {{{ Astrée
\section{Astrée}
\begin{quote}
Astrée stands for Analyseur statique
de logiciels temps-réel embarqués (real-time embedded software static
analyzer).

Astrée is a static program analyzer aiming at proving the absence of Run
Time Errors (RTE) in programs written in the C programming language.
On personal computers, such errors, commonly found in programs, usually
result in unpleasant error messages and the termination of the application,
and sometimes in a system crash. In embedded applications, such errors may
have graver consequences.

Astrée analyzes structured C programs, with complex memory usages, but
without dynamic memory allocation and recursion. This encompasses many
embedded programs as found in earth transportation, nuclear energy,
medical instrumentation, aeronautic, and aerospace applications, in
particular synchronous control/command such as electric flight control,
or space vessels maneuvers.
\end{quote}

I have never used this tool. It is a proprietary product, and the distributor
must be contacted in order to request a quote for the program.

\cite{astree}
% }}}
% {{{ BLAST
\section{BLAST}
\begin{quote}
BLAST (Berkeley Lazy Abstraction Software verification Tool) is a static
software verification tool for C language that solves the reachability
problem, i.e.\ whether a given program location can be reached from an
entry point (main function) by a valid execution.

Verification of safety properties may be reduced to the reachability,
and BLAST is used for such verification in the Linux Driver Verification
project.
\end{quote}

I have never used this tool. It is free software. The release notes
contain more information about the capabilities of this tool.

\cite{blast}
% }}}
% {{{ Clang Static Analyzer
\section{Clang Static Analyzer}
\begin{quote}
The Clang Static Analyzer is a source code analysis tool that finds bugs
in C, C++, and Objective-C programs.
\end{quote}

I have never used this tool, but it looks interesting. There is a web
browser interface to trace code paths.

\cite{clang}
% }}}
% {{{ Coccinelle
\section{Coccinelle}

\begin{quote}
Coccinelle is a program matching and transformation engine which
provides the language SmPL (Semantic Patch Language) for specifying
desired matches and transformations in C code. Coccinelle was initially
targeted towards performing collateral evolutions in Linux. Such
evolutions comprise the changes that are needed in client code in
response to evolutions in library APIs, and may include modifications
such as renaming a function, adding a function argument whose value is
somehow context-dependent, and reorganizing a data structure. Beyond
collateral evolutions, Coccinelle is successfully used (by us and others)
for finding and fixing bugs in systems code.
\end{quote}

I have never used this tool. It looks more complex than useful.

\cite{coccinelle}
% }}}
% {{{ Cppcheck
\section{Cppcheck}

\begin{quote}
Cppcheck is a static analysis tool for C/C++ code. Unlike C/C++ compilers
and many other analysis tools it does not detect syntax errors in the
code. Cppcheck primarily detects the types of bugs that the compilers
normally do not detect. The goal is to detect only real errors in the code
(i.e.\ have zero false positives).

Detect various kinds of bugs in your code:

\begin{itemize}
    \item Out of bounds checking
    \item Memory leaks checking
    \item Detect possible null pointer dereferences
    \item Check for uninitialized variables
    \item Check for invalid usage of STL
    \item Checking exception safety
    \item Warn if obsolete or unsafe functions are used
    \item Warn about unused or redundant code
    \item Detect various suspicious code indicating bugs
\end{itemize}
\end{quote}

I have never used this tool. It is free software, and looks useful.

\cite{cppcheck}
% }}}
% {{{ Eclair
\section{Eclair}

\begin{quote}
ECLAIR is a general platform for software verification. Applications
range from coding rule validation, to automatic generation of
test cases, to the proof of absence of run-time errors or generation
of counterexamples, and to the specification of code matchers and
rewriters based both syntactic and semantic conditions.
\end{quote}

I have never used this product. It is a proprietary product, and the
distributor must be contacted in order to request a quote for the program.

\cite{eclair}
% }}}
% {{{ Gdb
\section{Gdb}

\begin{quote}
GDB, the GNU Project debugger, allows you to see what is going on
``inside'' another program while it executes --- or what another program
was doing at the moment it crashed.

GDB can do four main kinds of things (plus other things in support
of these) to help you catch bugs in the act:

\begin{itemize}
    \item Start your program, specifying anything that might affect its behavior.
    \item Make your program stop on specified conditions.
    \item Examine what has happened, when your program has stopped.
    \item Change things in your program, so you can experiment
          with correcting the effects of one bug and go on to
          learn about another.
\end{itemize}
\end{quote}

This is a very useful program. It is free software.

\cite{gdb}
% }}}
% {{{ Indent
\section{Indent}

\begin{quote}
The indent program can be used to make code easier to read. It
can also convert from one style of writing C to another.\ indent
understands a substantial amount about the syntax of C, but it also
attempts to cope with incomplete and misformed syntax.
\end{quote}

This is a very useful program. It is free software.

\cite{indent}
% }}}
% {{{ MemorySanitizer
\section{MemorySanitizer}

\begin{quote}
MemorySanitizer (MSan) is a detector of uninitialized memory reads
in C/C++ programs.

Uninitialized values occur when stack- or heap-allocated memory
is read before it is written. MSan detects cases where such values
affect program execution.

MemorySanitizer is bit-exact: it can track uninitialized bits in
a bitfield. It will tolerate copying of uninitialized memory, and
also simple logic and arithmetic operations with it. In general,
MemorySanitizer silently tracks the spread of uninitialized data in
memory, and reports a warning when a code branch is taken (or not
taken) depending on an uninitialized value.
\end{quote}

\label{msan}
This, in combination with \hyperref[asan]{other} \hyperref[tsan]{sanitizers},
allows for powerful debugging capabilities.

\cite{msan}
% }}}
% {{{ Parasoft C/C++Test
\section{Parasoft C/C++Test}

\begin{quote}
Parasoft C/C++test is an integrated development testing solution for C and
C++. It automates a broad range of software quality practices—including
static code analysis, unit testing, code review, coverage analysis,
runtime error detection and more. It can be used in both host-based
and target-based code analysis and test flows, which is critical for
embedded and cross-platform development.
\end{quote}

I have never used this product. It is a proprietary product, and the
distributor must be contacted in order to request a quote for the program.

\cite{cpptest}
% }}}
% {{{ Splint
\section{Splint}

\begin{quote}
Splint is a tool for statically checking C programs for security
vulnerabilities and coding mistakes. With minimal effort, Splint can
be used as a better lint. If additional effort is invested adding
annotations to programs, Splint can perform stronger checking than
can be done by any standard lint.
\end{quote}

This is the spiritual successor to the well-known \texttt{lint} program.

\cite{splint}
% }}}
% {{{ ThreadSanitizer
\section{ThreadSanitizer}

\begin{quote}
ThreadSanitizer (aka TSan) is a data race detector for C/C++. Data
races are one of the most common and hardest to debug types of bugs
in concurrent systems. A data race occurs when two threads access
the same variable concurrently and at least one of the accesses
is write. C++11 standard officially bans data races as undefined
behavior.
\end{quote}

\label{tsan}
This, in combination with \hyperref[asan]{other} \hyperref[msan]{sanitizers},
allows for powerful debugging capabilities.

\cite{tsan}
% }}}
% {{{ Valgrind
\section{Valgrind}

\begin{quote}
Valgrind is an instrumentation framework for building dynamic analysis
tools. There are Valgrind tools that can automatically detect many
memory management and threading bugs, and profile your programs in
detail.

The Valgrind distribution currently includes six production-quality
tools: a memory error detector, two thread error detectors,
a cache and branch-prediction profiler, a call-graph generating
cache and branch-prediction profiler, and a heap profiler. It also
includes three experimental tools: a stack/global array overrun
detector, a second heap profiler that examines how heap blocks are
used, and a SimPoint basic block vector generator.
\end{quote}

This is a useful tool to detect leaks in applications with dynamically
allocated memory. The other tools included in the distribution are useful
as well.

\cite{valgrind}
% }}}
% {{{ Veracode
\section{Veracode}

\begin{quote}
SAST identifies critical vulnerabilities such as SQL injection,
cross-site scripting (XSS), buffer overflows, unhandled error
conditions and potential back-doors. Binary SAST technology
delivers actionable information that prioritizes flaws according
to severity and provides detailed remediation information to help
developers address them quickly.
\end{quote}

I have never used this product. It is a proprietary product, and the
distributor must be contacted in order to request a quote for the program.

\cite{sast}
% }}}
% {{{ Bibliography
\bibliographystyle{plain}
\begin{thebibliography}{9}

\bibitem{asan}
    \url{https://github.com/google/sanitizers/wiki/AddressSanitizer}
\bibitem{afl}
    \url{http://lcamtuf.coredump.cx/afl/}
\bibitem{astree}
    \url{http://www.astree.ens.fr/}
\bibitem{blast}
    \url{http://forge.ispras.ru/projects/blast}
\bibitem{clang}
    \url{http://clang-analyzer.llvm.org/}
\bibitem{coccinelle}
    \url{http://coccinelle.lip6.fr/}
\bibitem{cppcheck}
    \url{http://cppcheck.sourceforge.net/}
\bibitem{eclair}
    \url{http://bugseng.com/products/eclair}
\bibitem{gdb}
    \url{https://www.gnu.org/software/gdb/}
\bibitem{indent}
    \url{https://www.gnu.org/software/indent/}
\bibitem{msan}
    \url{https://github.com/google/sanitizers/wiki/MemorySanitizer}
\bibitem{cpptest}
    \url{https://www.parasoft.com/product/cpptest/}
\bibitem{splint}
    \url{http://splint.org/}
\bibitem{tsan}
    \url{https://github.com/google/sanitizers/wiki/ThreadSanitizerCppManual}
\bibitem{valgrind}
    \url{http://valgrind.org/}
\bibitem{sast}
    \url{https://www.veracode.com/products/binary-static-analysis-sast}

\end{thebibliography}
% }}}
\end{document}
